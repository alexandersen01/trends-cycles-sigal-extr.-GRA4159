\documentclass[10pt]{article}
\usepackage{tikz}
\usetikzlibrary{arrows.meta, positioning}

\usepackage{parskip}
\usepackage[margin=1in]{geometry} 
\usepackage{amsmath,amsthm,amssymb, graphicx, multicol, array}
\usepackage{enumitem}
\usepackage{amssymb}
\usepackage{float}
\usepackage{href-ul}
 
\newcommand{\N}{\mathbb{N}}
\newcommand{\Z}{\mathbb{Z}}
 
\newenvironment{problem}[2][Problem]{\begin{trivlist}
\item[\hskip \labelsep {\bfseries #1}\hskip \labelsep {\bfseries #2.}]}{\end{trivlist}}

\begin{document}
 
\title{Monetary policy: Inflation targeting in a closed economy pt. 2}
\author{Jakob Sverre Alexandersen\\
GRA4159 Trends, Cycles \& Signals Extraction\\
Lecture 4}
\maketitle

\tableofcontents
\newpage
\section{Inflation targeting and financial stability}
\begin{itemize}
    \item Two opposing views: 
    \begin{itemize}
        \item Central banks have to reconsider the desirability of (read: abandon) inflation targeting 
        \item Financial stability is not relevant for monetary policy
    \end{itemize}
    \item Woodford (2012): inflation targeting needs to be revised
    \item The CB cares about financial stability and wishes to set the interest rate in such a way that financial imbalances are as small as possible
    \item The loss function 
    \begin{gather*}
        L = \frac{1}{2}\left[(\pi - \pi^*)^2 + \lambda y^2 + \delta q^2\right]
    \end{gather*}
    $q $ is a financial variable measured as a deviation from its mean 
    \begin{itemize}
        \item High values of $q $ (think credit) gives ``high'' probability of a financial crisis / recession 
        \item Low values of $q $ is a sign that it is too hard to get credit 
        \item Woodford (2012): financial crises give loss to society and not only because it leads to a drop in economic activity
    \end{itemize}
\end{itemize}
\subsection{Economic mechanism}
\begin{itemize}
    \item IS and PC equations as before 
    \item The interest rate influences the size of financial imbalances: 
    \begin{gather*}
        q = -\phi(r - \rho) + w
    \end{gather*}
    where $w $ is a shock to financial imbalances
    \item We can write the equation above as 
    \begin{gather*}
        q = -\phi\left(r - \overset{\sim}{r}\right)
    \end{gather*}
    where $\overset{\sim}{r} = \rho + \frac{1}{\phi}w$ is the real rate that closes the financial gap
\end{itemize}
\newpage
\subsection{The monetary policy transmission mechanism}
A reduction in the key policy rate 

\begin{center}
    \begin{figure}[H]
        \centering
        \includegraphics[width=0.6\linewidth]{image.png}
    \end{figure}
\end{center}
\subsection{Optimal monetary policy}
\begin{itemize}
    \item The central bank minimizes the loss function subject to some equations
    \item The optimality condition becomes 
    \begin{gather*}
        (\pi - \pi^*) = -\frac{\lambda}{\gamma}y - \frac{\phi\delta}{\alpha\gamma}q\\
        y = -\frac{\gamma}{\lambda}(\pi - \pi^*) - \frac{\phi / \alpha}{\lambda / \delta} q
    \end{gather*}
    \item It can be written as: 
    \begin{gather*}
        (\pi - \pi^*) = -\frac{\lambda}{\gamma}y - \frac{\phi\delta}{\alpha\gamma}q\\
        = -\frac{\lambda + \delta\left(\frac{\phi}{\alpha}\right)^2}{\gamma}y - \frac{\phi\delta}{\gamma\alpha^2}(\alpha w - \phi v)
    \end{gather*}
\end{itemize}
\subsection{Graphical analysis}
\begin{itemize}
    \item To simplify, we assume $\pi^e = \pi^* $
    \item We will use two diagrams: 
    \begin{enumerate}
        \item $(y, \pi)$-diagram with two equations: 
        \begin{itemize}
            \item The Phillips curve (PC) with positive slope $\gamma $
            \item Monetary policy (MP) - with negative slope $-\frac{\lambda + \delta\left(\frac{\phi}{\alpha}\right)^2}{\gamma} $
        \end{itemize}
        \item $(y, r)$-diagram with one equation: 
        \begin{itemize}
            \item IS-equation solved wrt. the real interest rate
            \begin{gather*}
                r = \rho + \frac{1}{\alpha}v - \frac{1}{\alpha}y
            \end{gather*}
        \end{itemize}
    \end{enumerate}
\end{itemize}
\newpage 
\begin{figure}[h]
    \centering
    \begin{minipage}{0.48\textwidth}
      \centering
      \includegraphics[width=\linewidth]{image2.png}
    \end{minipage}\hfill
    \begin{minipage}{0.48\textwidth}
      \centering
      \includegraphics[width=\linewidth]{image3.png}
    \end{minipage}
\end{figure}
\subsection{A more realistic model}
\begin{itemize}
    \item Financial imbalances tend to build up in booms 
    \item Financial variables feed back to the level of activity and we get a financial accelerator 
    \item More specifically, we assume the following: 
    \begin{gather*}
        q = \tau y - \phi(r - \rho) + w\\
        y = -\alpha(r - \rho) + \chi q + v
    \end{gather*}
    while the rate of inflation is still given by the old equation
\end{itemize}
A reduction in the key policy rate 
\begin{center}
    \begin{figure}[H]
        \centering
        \includegraphics[width=0.6\linewidth]{image4.png}
    \end{figure}
\end{center}

\begin{itemize}
    \item Our equations can be rewritten as 
    \begin{gather*}
        y = -\frac{\alpha + \chi\phi}{1 - \chi\tau}(r - \bar{r}) = -\bar{\alpha}(r - \bar{r})\\
        q = -\frac{\tau\alpha + \phi}{1 - \chi\tau}\left(r - \overset{\sim}{r}\right) = -\bar{\phi}\left(r - \overset{\sim}{r}\right)
    \end{gather*}
    where $\bar{r} = \rho + \frac{1}{\alpha + \chi\tau}v + \frac{\chi}{\alpha + \chi\tau}w \text{ and } \overset{\sim}{r} = \rho + \frac{\tau}{\tau\alpha + \phi}v + \frac{1}{\tau\alpha + \phi}w $
    \item The optimality condition becomes 
    \begin{gather*}
        (\pi - \pi^*) = -\frac{\lambda}{\gamma}y - \frac{\bar{\phi}\delta}{\bar{\alpha}\gamma}q
    \end{gather*}
\end{itemize}


\end{document}
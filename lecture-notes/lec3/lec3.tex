\documentclass[10pt]{article}

\usepackage{parskip}
\usepackage[margin=1in]{geometry} 
\usepackage{amsmath,amsthm,amssymb, graphicx, multicol, array}
\usepackage{enumitem}
\usepackage{amssymb}
\usepackage{float}
\usepackage{href-ul}
 
\newcommand{\N}{\mathbb{N}}
\newcommand{\Z}{\mathbb{Z}}
 
\newenvironment{problem}[2][Problem]{\begin{trivlist}
\item[\hskip \labelsep {\bfseries #1}\hskip \labelsep {\bfseries #2.}]}{\end{trivlist}}

\begin{document}
 
\title{Monetary policy: Inflation targeting in a closed economy}
\author{Jakob Sverre Alexandersen\\
GRA4159 Trends, Cycles \& Signals Extraction\\
Lecture 3}
\maketitle

\tableofcontents
\newpage
\section{How does monetary policy work?}
The policy rate is the most important tool for the central bank 
\begin{itemize}
    \item In Norway: the interest rate on large banks overnight deposits at Norges Bank (up to a quota)
    \item Createas the floor for market interest rate 
    \item Affects the rate of inflation and other key macroeconomic variables through different channels 
    \item Works through several channels
    \begin{itemize}
        \item Exchange rate channel
        \item Expectations channel
        \item Demand channel
    \end{itemize}
\end{itemize}
\section{Monetary policy trade offs}
Critera for an appropriate interest rate path:
\begin{enumerate}
    \item The inflation target is achieved: the interest rate path should stabilize inflation at target or bring inflation back to target after a deviation has occured 
    \item The inflation targeting regime is flexible: The interest rate path should provide a reasonable balance between the path for inflation and the path for capacity utilization in the economy 
    \item Monetary poly is robust: the interest rate should be set so that monetary policy mitigates the risk of a buildup of financial imbalances, and so that acceptable developments in inflation and output are also the likely outcome under alternative assumptions about the functioning of the economy
\end{enumerate}
\section{Monetary policy under inflation targeting}
\textbf{transmission mechanism}
\begin{itemize}
    \item the output gap 
    \begin{gather*}
        y = \alpha(i - \pi^e - \rho) + v
    \end{gather*}
    where $y = \frac{Y - \bar{Y}}{\bar{Y}} $ denotes the output gap and the real interest rate is $r = i - \pi^e $
    \item alternatively, we can write the IS-equation as 
    \begin{gather*}
        y = \alpha(r - \bar{r})
    \end{gather*}
    where $\bar{r} = \rho + \frac{1}{\alpha}v $ is the short-run neutral real rate 
    \item inflation equation:
    \begin{gather*}
        \pi = \pi^e + \gamma y + u
    \end{gather*}
    where $u $ is a cost-push shock (or inflation shock)
\end{itemize}
\newpage
\textbf{monetary policy}
\begin{itemize}
    \item loss function: 
    \begin{gather*}
        L = \frac{1}{2}\left[(\pi - \pi^*)^2 + \lambda y^2\right]
    \end{gather*}
    \item parameter $\lambda $ measures how much weight the central bank assigns to production stability relative to price stability
    \begin{itemize}
        \item $\lambda = 0 $: strict inflation targeting 
        \item $\lambda > 0 $: flexible inflation targeting 
        \item $\lambda = \infty $: output gap targeting 
    \end{itemize}
\end{itemize}
\textbf{optimal monetary policy}
\begin{itemize}
    \item minimize the loss function, given the economic mechanisms described above: 
    \begin{gather*}
        \min_r \frac{1}{2}\left[(\pi - \pi^*)^2 + \lambda y^2\right]
    \end{gather*}
    \item the FOC can be written as: 
    \begin{gather*}
        \pi - \pi^* = \frac{-\lambda}{\gamma}y \iff y = -\frac{\gamma}{\lambda}(\pi - \pi^*)
    \end{gather*}
    the latter shows the extent to which the central bank is willing to drive the economy into a recession when inflation is above the target
\end{itemize}

% \begin{center}
%     \begin{figure}[H]
%         \centering
%         \includegraphics[width=0.8\linewidth]{image.png}
%     \end{figure}
% \end{center}


\end{document}